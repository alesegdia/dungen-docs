%%%%%%%%%%%%%%%%%%%%%%%%%%%%%%%%%%%%%%%%%%%%%%%%%%%%%%%%%%%%%%%%%%%%%
%%                                                                 							%%
%%		Trabajo Fin de GRADO		                       				%%
%%		TITULO 																			%%
%%		AUTOR												%%
%%                                                                 							%%
%%%%%%%%%%%%%%%%%%%%%%%%%%%%%%%%%%%%%%%%%%%%%%%%%%%%%%%%%%%%%%%%%%%%%

%%  Include con la definicion de estilos por el usuario
%%%%%%%%%%%%%%%%%%%%%%%%%%%%%%%%%%%%%%%%%%%%%%%%%%%%%%%%%%%%%%%%%%%%%

\input definl

%%  Paqueteria necesaria de fabrica
%%%%%%%%%%%%%%%%%%%%%%%%%%%%%%%%%%%%%%%%%%%%%%%%%%%%%%%%%%%%%%%%%%%%%
\usepackage{hyperref}
\usepackage{palatino}
\usepackage[dvips]{graphicx} 	  % para importar combinados latex
\usepackage{color}          		      % para importar dibujos coloreados
\usepackage{rotating}        		  % para usar \begin{sideways} que rota tabla 90 grados 
\usepackage{epsfig}          		  % para rotar figuras de Xfig  poniendo % \begin{sideways} 
\usepackage{amsmath}         	  % para usar matrix y pmatrix environment
\usepackage{stmaryrd}        		  % para usar la \bigsqcap
\usepackage{verbatim}        		  % para poner salidas de pantallas
\usepackage{listings}       	 	  % para imprimir codigo fuente
\usepackage{shortvrb}				  % 
\usepackage{url}						  % 
\usepackage{subfigure}			  % 
\usepackage[usenames,dvipsnames]{xcolor}

  

%%%%%%%%%%%%%%%%%%%%%%%%%%%%%%%%%%%%%%%%%%%%%%%%%%%%%%%%%%%%%%%%%%%%%
%%  Configuracion de paquetes
%%%%%%%%%%%%%%%%%%%%%%%%%%%%%%%%%%%%%%%%%%%%%%%%%%%%%%%%%%%%%%%%%%%%%
\renewcommand\lstlistingname{Listado}                   %  default is Listing
%\renewcommand\lstlistlistingname{\'Indice de listados}  %  default is Listings 
%\renewcommand\thelstlisting{\thechapter .\arabic{lstlisting}} % captionstyle


\lstset{
  language=Java,
  stepnumber=1}

\usepackage{listings}
\usepackage{courier}
\lstset{
         basicstyle=\scriptsize\ttfamily, % Standardschrift
         numbers=left,               % Ort der Zeilennummern
         numberstyle=\scriptsize,          % Stil der Zeilennummern
         %stepnumber=2,               % Abstand zwischen den Zeilennummern
         numbersep=5pt,              % Abstand der Nummern zum Text
         tabsize=2,                  % Groesse von Tabs
         extendedchars=true,         %
         breaklines=true,            % Zeilen werden Umgebrochen
         keywordstyle=\bfseries\color{blue},
            frame=b,
 %        keywordstyle=[1]\textbf,    % Stil der Keywords
 %        keywordstyle=[2]\textbf,    %
 %        keywordstyle=[3]\textbf,    %
 %        keywordstyle=[4]\textbf,   \sqrt{\sqrt{}} %
         stringstyle=\color{white}\ttfamily, % Farbe der String
         showspaces=false,           % Leerzeichen anzeigen ?
         showtabs=false,             % Tabs anzeigen ?
         xleftmargin=17pt,
         framexleftmargin=17pt,
         framexrightmargin=5pt,
         framexbottommargin=4pt,
		 emph={},
		 emphstyle={\color{black}\bfseries},
         %backgroundcolor=\color{lightgray},
         showstringspaces=false      % Leerzeichen in Strings anzeigen ?        
}

%\captionsetup[lstlisting]{singlelinecheck=false, labelfont={blue}, textfont={blue}}
\usepackage{caption}
\DeclareCaptionFont{white}{\color{white}}
\DeclareCaptionFormat{listing}{\colorbox[cmyk]{1, 1, 1,0.01}{\parbox{\textwidth}{\hspace{15pt}#1#2#3}}}
\captionsetup[lstlisting]{format=listing,labelfont=white,textfont=white, singlelinecheck=false, margin=0pt, font={bf,footnotesize}}


%%  Configuraciones varias
%%%%%%%%%%%%%%%%%%%%%%%%%%%%%%%%%%%%%%%%%%%%%%%%%%%%%%%%%%%%%%%%%%%%%
\newcommand{\corregir}{\color{blue}}  	 	% pintar en color azul
\setlength{\parskip}{2ex}              			% despues del parrafo, doble linea


% Para que no aparezcan las cabeceras de las páginas que están en blanco
%%%%%%%%%%%%%%%%%%%%%%%%%%%%%%%%%%%%%%%%%%%%%%%%%%%%%%%%%%%%%%%%%%%%%
\makeatletter 
\def\cleardoublepage{\clearpage\if@twoside \ifodd\c@page\else
  \hbox{} 
  \thispagestyle{empty} 
  \newpage
  \if@twocolumn\hbox{}\newpage\fi\fi\fi} 
\makeatother

%%  Cortes de palabras especiales
%%%%%%%%%%%%%%%%%%%%%%%%%%%%%%%%%%%%%%%%%%%%%%%%%%%%%%%%%%%%%%%%%%%%%
\hyphenation{
ejem-plo Algo-ritmo
}


%\includeonly{titulo, prologo,intro,capitulo1}


\begin{document} 

%%  Titulo e Indices
%%%%%%%%%%%%%%%%%%%%%%%%%%%%%%%%%%%%%%%%%%%%%%%%%%%%%%%%%%%%%%%%%%%%%%
\pagenumbering{roman}
\thispagestyle{empty}

{


\thispagestyle{empty}
\begin{center}
\includegraphics[scale=.6]{img/logouhu}
\end{center}

\vspace*{0cm}
\Large 

\begin{center}

{\normalsize \sc Escuelta Técnica Superior de Ingeniería \\ de la Universidad de Huelva}

{\large \bf Grado en Ingeniería Informática}

\vspace*{1.5cm}

{\sc Trabajo Fin de Grado}


{\LARGE \bf Generación procedimental \\de mapas de tiles 2D. \\ Análisis e investigación del problema. }

\end{center} 




\vspace*{0.5cm}
%\vfill

\begin{center}
{\normalsize Autor: \\ {\bf Alejandro Seguí Díaz}}
\end{center}

\begin{center}
%{\footnotesize Tutores:}
{\small Tutores: }
\vspace*{0.2cm}
{\small \\  Gonzalo A. Aranda Corral \\ Daniel Márquez Quintanilla}
\end{center}

\vspace*{0.5cm}
\vfill
\begin{center}
{\footnotesize Huelva, 30 de junio de 2015.\\Curso académico 2014/15.}
\end{center}





\newpage
\thispagestyle{empty}
\mbox{ }
\newpage
\thispagestyle{empty}

\newpage
\thispagestyle{empty}

\mbox{ }

\vfill

\begin{flushright}
  \begin{minipage}{9cm}

\end{minipage}
\end{flushright}

\vfill

\newpage
\thispagestyle{empty}
\mbox{ }

}


\clearpage
\pagestyle{plain}
\tableofcontents
\clearpage
\listoffigures

%%  Contenido del trabajo
%%%%%%%%%%%%%%%%%%%%%%%%%%%%%%%%%%%%%%%%%%%%%%%%%%%%%%%%%%%%%%%%%%%%%%
\pagenumbering{arabic}
\pagestyle{fancy}

%\chapter*{Prólogo}
\label{intro:prologo}
\addcontentsline{toc}{chapter}{Prólogo}


\chapter*{Introducción}
\label{intro:intro}
\addcontentsline{toc}{section}{Introducción.}

Lorem ipsum dolor sit amet, consectetur adipiscing elit. Curabitur tincidunt vitae leo eget vestibulum. Donec non mauris ultrices, accumsan purus vel, tincidunt nisl. Vivamus vehicula risus maximus lacus bibendum, at vestibulum urna tempus. Ut ac orci est. Sed tempor varius nibh nec laoreet. Duis convallis diam vitae aliquet semper. Sed ultricies ante nec erat cursus, sed sollicitudin velit scelerisque. Vestibulum eu faucibus risus. Donec interdum vehicula ullamcorper. Donec quis orci dapibus, blandit ipsum et, interdum erat. Aliquam sit amet urna semper, condimentum lacus vel, viverra velit. In hac habitasse platea dictumst. Aenean hendrerit maximus dignissim. Proin at nulla urna.

Nunc viverra volutpat bibendum. Nunc augue orci, tempus nec interdum sit amet, ornare id elit. Integer congue risus vitae ipsum pharetra rutrum. Curabitur mollis sagittis pretium. Etiam a lacus sed mauris rhoncus ullamcorper non sit amet felis. Ut arcu ante, sodales eu sapien ullamcorper, tempus facilisis enim. Quisque sit amet mauris pharetra erat porta rhoncus nec interdum odio. Nulla lectus metus, placerat id blandit at, volutpat non velit. Nam non viverra dolor, sed vulputate orci. Fusce non enim fermentum, varius libero vel, viverra nulla. Mauris mollis lectus non dolor blandit, non consectetur dolor blandit.

Proin porttitor interdum risus pharetra condimentum. Aenean et mauris ligula. Aenean facilisis odio ut metus consequat varius. Aliquam erat volutpat. In hac habitasse platea dictumst. Vestibulum sit amet velit urna. Donec ac volutpat sem, sit amet fermentum risus. Ut eu augue sit amet est vulputate maximus. Aliquam vitae pellentesque quam. Maecenas vel velit ut dui finibus dapibus. Vivamus consectetur hendrerit consequat. Proin id justo vitae nulla aliquet eleifend et quis enim. Quisque bibendum sit amet neque eu sollicitudin.



\section*{Videojuegos}
\addcontentsline{toc}{section}{Videojuegos}

Lorem ipsum dolor sit amet, consectetur adipiscing elit. Curabitur tincidunt vitae leo eget vestibulum. Donec non mauris ultrices, accumsan purus vel, tincidunt nisl. Vivamus vehicula risus maximus lacus bibendum, at vestibulum urna tempus. Ut ac orci est. Sed tempor varius nibh nec laoreet. Duis convallis diam vitae aliquet semper. Sed ultricies ante nec erat cursus, sed sollicitudin velit scelerisque. Vestibulum eu faucibus risus. Donec interdum vehicula ullamcorper. Donec quis orci dapibus, blandit ipsum et, interdum erat. Aliquam sit amet urna semper, condimentum lacus vel, viverra velit. In hac habitasse platea dictumst. Aenean hendrerit maximus dignissim. Proin at nulla urna.

Nunc viverra volutpat bibendum. Nunc augue orci, tempus nec interdum sit amet, ornare id elit. Integer congue risus vitae ipsum pharetra rutrum. Curabitur mollis sagittis pretium. Etiam a lacus sed mauris rhoncus ullamcorper non sit amet felis. Ut arcu ante, sodales eu sapien ullamcorper, tempus facilisis enim. Quisque sit amet mauris pharetra erat porta rhoncus nec interdum odio. Nulla lectus metus, placerat id blandit at, volutpat non velit. Nam non viverra dolor, sed vulputate orci. Fusce non enim fermentum, varius libero vel, viverra nulla. Mauris mollis lectus non dolor blandit, non consectetur dolor blandit.



\section*{Motivación y Objetivos.}
\addcontentsline{toc}{section}{Motivación y Objetivos}

lorem ipsum dolor sit amet, consectetur adipiscing elit. curabitur tincidunt vitae leo eget vestibulum. donec non mauris ultrices, accumsan purus vel, tincidunt nisl. vivamus vehicula risus maximus lacus bibendum, at vestibulum urna tempus. ut ac orci est. sed tempor varius nibh nec laoreet. duis convallis diam vitae aliquet semper. sed ultricies ante nec erat cursus, sed sollicitudin velit scelerisque. vestibulum eu faucibus risus. donec interdum vehicula ullamcorper. donec quis orci dapibus, blandit ipsum et, interdum erat. aliquam sit amet urna semper, condimentum lacus vel, viverra velit. in hac habitasse platea dictumst. aenean hendrerit maximus dignissim. proin at nulla urna.

Nunc viverra volutpat bibendum. Nunc augue orci, tempus nec interdum sit amet, ornare id elit. Integer congue risus vitae ipsum pharetra rutrum. Curabitur mollis sagittis pretium. Etiam a lacus sed mauris rhoncus ullamcorper non sit amet felis. Ut arcu ante, sodales eu sapien ullamcorper, tempus facilisis enim. Quisque sit amet mauris pharetra erat porta rhoncus nec interdum odio. Nulla lectus metus, placerat id blandit at, volutpat non velit. Nam non viverra dolor, sed vulputate orci. Fusce non enim fermentum, varius libero vel, viverra nulla. Mauris mollis lectus non dolor blandit, non consectetur dolor blandit.

Proin porttitor interdum risus pharetra condimentum. Aenean et mauris ligula. Aenean facilisis odio ut metus consequat varius. Aliquam erat volutpat. In hac habitasse platea dictumst. Vestibulum sit amet velit urna. Donec ac volutpat sem, sit amet fermentum risus. Ut eu augue sit amet est vulputate maximus. Aliquam vitae pellentesque quam. Maecenas vel velit ut dui finibus dapibus. Vivamus consectetur hendrerit consequat. Proin id justo vitae nulla aliquet eleifend et quis enim. Quisque bibendum sit amet neque eu sollicitudin.




\section*{Solución Propuesta.}
\addcontentsline{toc}{section}{Solución Propuesta}


En el capítulo~\ref{cap:capitulo1} lorem ipsum dolor sit amet, consectetur adipiscing elit. curabitur tincidunt vitae leo eget vestibulum. donec non mauris ultrices, accumsan purus vel, tincidunt nisl. vivamus vehicula risus maximus lacus bibendum, at vestibulum urna tempus. ut ac orci est. sed tempor varius nibh nec laoreet. duis convallis diam vitae aliquet semper. sed ultricies ante nec erat cursus, sed sollicitudin velit scelerisque. vestibulum eu faucibus risus. donec interdum vehicula ullamcorper. donec quis orci dapibus, blandit ipsum et, interdum erat. aliquam sit amet urna semper, condimentum lacus vel, viverra velit. in hac habitasse platea dictumst. aenean hendrerit maximus dignissim. proin at nulla urna.


Nunc viverra volutpat bibendum. Nunc augue orci, tempus nec interdum sit amet, ornare id elit. Integer congue risus vitae ipsum pharetra rutrum. Curabitur mollis sagittis pretium. Etiam a lacus sed mauris rhoncus ullamcorper non sit amet felis. Ut arcu ante, sodales eu sapien ullamcorper, tempus facilisis enim. Quisque sit amet mauris pharetra erat porta rhoncus nec interdum odio. Nulla lectus metus, placerat id blandit at, volutpat non velit. Nam non viverra dolor, sed vulputate orci. Fusce non enim fermentum, varius libero vel, viverra nulla. Mauris mollis lectus non dolor blandit, non consectetur dolor blandit.

Proin porttitor interdum risus pharetra condimentum. Aenean et mauris ligula. Aenean facilisis odio ut metus consequat varius. Aliquam erat volutpat. In hac habitasse platea dictumst. Vestibulum sit amet velit urna. Donec ac volutpat sem, sit amet fermentum risus. Ut eu augue sit amet est vulputate maximus. Aliquam vitae pellentesque quam. Maecenas vel velit ut dui finibus dapibus. Vivamus consectetur hendrerit consequat. Proin id justo vitae nulla aliquet eleifend et quis enim. Quisque bibendum sit amet neque eu sollicitudin.




\section*{Estructura de la memoria.}
\addcontentsline{toc}{section}{Estructura de la memoria}

Esta memoria se estructura en varios capítulos, con la siguiente distribución de los temas trabajados:

\begin{itemize}
	\item Capítulo~\ref{cap:capitulo1}. Nunc viverra volutpat bibendum. Nunc augue orci, tempus nec interdum sit amet, ornare id elit. Integer congue risus vitae ipsum pharetra rutrum. Curabitur mollis sagittis pretium. Etiam a lacus sed mauris rhoncus ullamcorper non sit amet felis. Ut arcu ante, sodales eu sapien ullamcorper, tempus facilisis enim. Quisque sit amet mauris pharetra erat porta rhoncus nec interdum odio. Nulla lectus metus, placerat id blandit at, volutpat non velit. Nam non viverra dolor, sed vulputate orci. Fusce non enim fermentum, varius libero vel, viverra nulla. Mauris mollis lectus non dolor blandit, non consectetur dolor blandit.
	\item Capítulo~\ref{cap:capitulo2}. Nunc viverra volutpat bibendum. Nunc augue orci, tempus nec interdum sit amet, ornare id elit. Integer congue risus vitae ipsum pharetra rutrum. Curabitur mollis sagittis pretium. Etiam a lacus sed mauris rhoncus ullamcorper non sit amet felis. Ut arcu ante, sodales eu sapien ullamcorper, tempus facilisis enim. Quisque sit amet mauris pharetra erat porta rhoncus nec interdum odio. Nulla lectus metus, placerat id blandit at, volutpat non velit. Nam non viverra dolor, sed vulputate orci. Fusce non enim fermentum, varius libero vel, viverra nulla. Mauris mollis lectus non dolor blandit, non consectetur dolor blandit.
	\item Capítulo~\ref{cap:capitulo3}. Nunc viverra volutpat bibendum. Nunc augue orci, tempus nec interdum sit amet, ornare id elit. Integer congue risus vitae ipsum pharetra rutrum. Curabitur mollis sagittis pretium. Etiam a lacus sed mauris rhoncus ullamcorper non sit amet felis. Ut arcu ante, sodales eu sapien ullamcorper, tempus facilisis enim. Quisque sit amet mauris pharetra erat porta rhoncus nec interdum odio. Nulla lectus metus, placerat id blandit at, volutpat non velit. Nam non viverra dolor, sed vulputate orci. Fusce non enim fermentum, varius libero vel, viverra nulla. Mauris mollis lectus non dolor blandit, non consectetur dolor blandit.
	\item Capítulo~\ref{cap:capitulo4}. Nunc viverra volutpat bibendum. Nunc augue orci, tempus nec interdum sit amet, ornare id elit. Integer congue risus vitae ipsum pharetra rutrum. Curabitur mollis sagittis pretium. Etiam a lacus sed mauris rhoncus ullamcorper non sit amet felis. Ut arcu ante, sodales eu sapien ullamcorper, tempus facilisis enim. Quisque sit amet mauris pharetra erat porta rhoncus nec interdum odio. Nulla lectus metus, placerat id blandit at, volutpat non velit. Nam non viverra dolor, sed vulputate orci. Fusce non enim fermentum, varius libero vel, viverra nulla. Mauris mollis lectus non dolor blandit, non consectetur dolor blandit.
	\item Capítulo~\ref{cap:capitulo5}. Nunc viverra volutpat bibendum. Nunc augue orci, tempus nec interdum sit amet, ornare id elit. Integer congue risus vitae ipsum pharetra rutrum. Curabitur mollis sagittis pretium. Etiam a lacus sed mauris rhoncus ullamcorper non sit amet felis. Ut arcu ante, sodales eu sapien ullamcorper, tempus facilisis enim. Quisque sit amet mauris pharetra erat porta rhoncus nec interdum odio. Nulla lectus metus, placerat id blandit at, volutpat non velit. Nam non viverra dolor, sed vulputate orci. Fusce non enim fermentum, varius libero vel, viverra nulla. Mauris mollis lectus non dolor blandit, non consectetur dolor blandit.
	\item Capítulo~\ref{cap:capitulo6}. Nunc viverra volutpat bibendum. Nunc augue orci, tempus nec interdum sit amet, ornare id elit. Integer congue risus vitae ipsum pharetra rutrum. Curabitur mollis sagittis pretium. Etiam a lacus sed mauris rhoncus ullamcorper non sit amet felis. Ut arcu ante, sodales eu sapien ullamcorper, tempus facilisis enim. Quisque sit amet mauris pharetra erat porta rhoncus nec interdum odio. Nulla lectus metus, placerat id blandit at, volutpat non velit. Nam non viverra dolor, sed vulputate orci. Fusce non enim fermentum, varius libero vel, viverra nulla. Mauris mollis lectus non dolor blandit, non consectetur dolor blandit.
	\item Capítulo~\ref{cap:capitulo7}. Nunc viverra volutpat bibendum. Nunc augue orci, tempus nec interdum sit amet, ornare id elit. Integer congue risus vitae ipsum pharetra rutrum. Curabitur mollis sagittis pretium. Etiam a lacus sed mauris rhoncus ullamcorper non sit amet felis. Ut arcu ante, sodales eu sapien ullamcorper, tempus facilisis enim. Quisque sit amet mauris pharetra erat porta rhoncus nec interdum odio. Nulla lectus metus, placerat id blandit at, volutpat non velit. Nam non viverra dolor, sed vulputate orci. Fusce non enim fermentum, varius libero vel, viverra nulla. Mauris mollis lectus non dolor blandit, non consectetur dolor blandit.
	\item Capítulo~\ref{cap:capitulo8}. Nunc viverra volutpat bibendum. Nunc augue orci, tempus nec interdum sit amet, ornare id elit. Integer congue risus vitae ipsum pharetra rutrum. Curabitur mollis sagittis pretium. Etiam a lacus sed mauris rhoncus ullamcorper non sit amet felis. Ut arcu ante, sodales eu sapien ullamcorper, tempus facilisis enim. Quisque sit amet mauris pharetra erat porta rhoncus nec interdum odio. Nulla lectus metus, placerat id blandit at, volutpat non velit. Nam non viverra dolor, sed vulputate orci. Fusce non enim fermentum, varius libero vel, viverra nulla. Mauris mollis lectus non dolor blandit, non consectetur dolor blandit.
	\item Bibliografía.
\end{itemize}

\chapter{Estado del arte.}\label{cap:capitulo1}

\section{Generación de contenido procedimental.}



\section{Generación de mapas.}



   % 
\chapter{Especificaciones.}\label{cap:capitulo2}

En este capítulo definiremos las especificaciones y directrices tanto del tipo de juego como del sistema de generación. Dichas directrices que delimitarán el sistema a crear, han sido establecidas por la empresa indie de videojuegos \emph{TheGameKitchen}, y el sistema será empleado en un futuro título de la misma.

\section{Tipo de juego.}

El género de juego al que nos enfocaremos, será del tipo \emph{roguelike} \cite{rlike}, cuyas características principales son:

\begin{itemize}
	\item \emph{Generación de mazmorras}. Cada vez que el jugador inicia una partida, la experiencia será ligeramente distinta.
	\item \emph{Importancia considerable a la exploración}. El hecho de que las mazmorras no sean siempre iguales, incita al jugador a tener que invertir tiempo en explorar para poder encontrar la salida.
	\item \emph{Desarrollo del juego por plantas}. El objetivo del jugador suele ser llegar a una habitación considerada como final, donde puede elegir pasar a una siguiente planta, o investigar un poco más en la presente.
	\item \emph{Dificultad progresiva}. Cada planta, tendrá una dificultad ligeramente mayor a la de la anterior hasta llegar a la última.
	\item \emph{Muerte permanente}. Una vez que el jugador muere, no hay manera de cargar la partida. La unica opción es comenzar de nuevo.
\end{itemize}

Históricamente, el género \emph{roguelike} se identificaba además por otro tipo de características, como la mecánica por turnos o el énfasis en jugabilidad y desinterés en los gráficos, pero con el tiempo, el género se ha ido abriendo paso a una definición más genérica. Debido a ésto, hoy en día existen desde \emph{shooters} considerados \emph{roguelike}, como por ejemplo \emph{Tower of Guns} o \emph{Paranautical Activity}, hasta \emph{plataformas}, como \emph{Risk of Rain} o \emph{Spelunky}.

En concreto, el juego estará representado en un mapa de tiles.


\section{Directrices para la generación.}

   % 
\chapter{Representación.}\label{cap:capitulo3}




\section{Topología.}



\section{Habitaciones.}


\subsection{Puertas potenciales.}

\subsection{Prefabs.}

\subsection{Instancias.}

\subsection{Mapa.}

   % 
\chapter{Movimientos.}\label{cap:capitulo4}



\section{Cómputo de posibles movimientos.}


\section{Conexiones entre mapa y habitaciones.}



   % 
\chapter{Sistema de generación.}\label{cap:capitulo5}



\section{Interfaz de construcción.}


\section{Guardado de movimientos}

\section{Interfaz aleatoria.}


\section{Interfaz basada en búsqueda.}

   % 
\chapter{Experimentación.}\label{cap:capitulo6}



\section{Seccion31.}


\section{Seccion32.}



   % 
\chapter{Experimentación.}\label{cap:capitulo7}


\section{Interfaz de experimentación.}


\section{Flexibilidad y posibilidades.}


\section{Eficiencia}

Recordemos que, aunque no se restringía un tamaño al escenario, se estima como tamaño máximo un escenario de 64x64 tiles. Aún así, para estresar el algoritmo, se han elaborado mapas con tamaños mayores. Por ejemplo, en uno de ellos el tamaño de habitación de 20x20, y en el mejor caso, con habitaciones de este tamaño cuadradas, podríamos tener 9 en un mapa de 64x64, pero veremos tiempos bastante buenos con una generación de este tipo hasta para escenarios de 30 habitaciones.

Como no se ha implementado en móvil, se ha considerado que un tiempo es bueno, cuando es menor de un segundo. Si es menor de dos segundos, también se considerará como aceptable. Aún así, habría que hacer una prueba real, que debido a que Java es la plataforma de elección para el desarrollo a Android, no sería complejo. En un caso real de un juego de móvil además, se presupone menor complejidad debido al sistema donde se ejecuta, y se puede entender que un tiempo aceptable de espera a la generación es hasta 10 segundos, y en esto se fundamenta la aceptación de menor de dos segundos como solución buena.

Destacar que las mediciones realizadas para comprobar el impacto de algunos parámetros se han usado sin tener en cuenta los demás. Al final, podremos ver una medición realizada sobre una configuración que se ha conseguido con tiempos muy buenos y calidad notable.

Las pruebas se han realizado en un portátil HP Compaq Presario V6000 con las siguientes características:

\begin{center}
	\begin{tabular}{ | c | c | }
\hline
Modelo procesador & AMD Athlon 64 X2 \\ 
Núcleos procesador & 2 (1 en uso) \\
Velocidad procesador & 1.7 Ghz \\
Caché procesador & 512 KB L2 \\
Memoria RAM & 1 GB \\
\hline
	\end{tabular}
\end{center}


\subsection{Influencia del factor selector de movimientos}

\begin{center}
	\begin{tabular}{ | c | c | }
\hline
 		Property & Value \\ \hline
DoorGenType & ALL \\ 
SolverType & BestSearch \\ 
CacheType & NO CACHE \\ 
\hline
	\end{tabular}
\end{center}

\begin{center}
	\begin{tabular}{ | c | c | c | c | c | c | c | }
\hline
Tam. habs. & Modelos & Inst./modelo & Total habs. & 0.75f & 0.85f & 0.95f\\ \hline 
10 & 2 & 20 & 40 & 0.1763 & 0.1967 & 0.3667 \\ 
10 & 2 & 30 & 60 & 0.4945 & 0.6254 & 1.4392 \\ 
10 & 4 & 20 & 80 & 1.2336 & 1.6429 & 4.0735 \\ 
10 & 4 & 30 & 120 & 4.8805 & 6.6694 & 18.1853 \\ 
\hline
	\end{tabular}
\end{center}


\subsection{Ejemplo real}

bs-opt-medium-sample
\begin{center}
	\begin{tabular}{ | c | c | }
\hline
 		Property & Value \\ \hline
DoorGenType & RANDOM \\ 
Refresher cache divisor & 10 \\ 
SolverType & BestSearch \\ 
BestSearch DPE Divisor & 0.9 \\ 
CacheType & REFRESHER \\ 
Random doors param & 0.5 \\ 
\hline
	\end{tabular}
\end{center}

\begin{center}
	\begin{tabular}{ | c | c | c | c | c | }
\hline
Tam. habs. & Modelos & Instancias/modelo & Total habs. & Tiempo \\ \hline 
10 & 5 & 1 & 5 & 0.0035 \\ 
10 & 5 & 2 & 10 & 0.0127 \\ 
10 & 5 & 3 & 15 & 0.0246 \\ 
10 & 10 & 1 & 10 & 0.0257 \\ 
10 & 10 & 2 & 20 & 0.1216 \\ 
10 & 10 & 3 & 30 & 0.1667 \\ 
10 & 15 & 1 & 15 & 0.1143 \\ 
10 & 15 & 2 & 30 & 0.3776 \\ 
10 & 15 & 3 & 45 & 0.8163 \\ 
\hline
	\end{tabular}
\end{center}

bs-opt-real-sample
\begin{center}
	\begin{tabular}{ | c | c | }
\hline
 		Property & Value \\ \hline
DoorGenType & RANDOM \\ 
Refresher cache divisor & 10 \\ 
SolverType & BestSearch \\ 
BestSearch DPE Divisor & 0.9 \\ 
CacheType & REFRESHER \\ 
Random doors param & 0.5 \\ 
\hline
	\end{tabular}
\end{center}

\begin{center}
	\begin{tabular}{ | c | c | c | c | c | }
\hline
Tam. habs. & Modelos & Instancias/modelo & Total habs. & Tiempo \\ \hline 
6 & 5 & 1 & 5 & 0.0007 \\ 
6 & 5 & 2 & 10 & 0.0029 \\ 
6 & 5 & 3 & 15 & 0.0073 \\ 
6 & 5 & 4 & 20 & 0.0168 \\ 
6 & 10 & 1 & 10 & 0.0050 \\ 
6 & 10 & 2 & 20 & 0.0266 \\ 
6 & 10 & 3 & 30 & 0.0679 \\ 
6 & 10 & 4 & 40 & 0.1601 \\ 
6 & 15 & 1 & 15 & 0.0157 \\ 
6 & 15 & 2 & 30 & 0.0905 \\ 
6 & 15 & 3 & 45 & 0.3026 \\ 
6 & 15 & 4 & 60 & 0.6859 \\ 
6 & 20 & 1 & 20 & 0.0398 \\ 
6 & 20 & 2 & 40 & 0.2525 \\ 
6 & 20 & 3 & 60 & 0.7766 \\ 
6 & 20 & 4 & 80 & 2.0803 \\ 
\hline
	\end{tabular}
\end{center}


\subsection{Muchos tipos de habitación}

bs-opt-variabilty-sample
\begin{center}
	\begin{tabular}{ | c | c | }
\hline
 		Property & Value \\ \hline
DoorGenType & RANDOM \\ 
Refresher cache divisor & 10 \\ 
SolverType & BestSearch \\ 
BestSearch DPE Divisor & 0.9 \\ 
CacheType & REFRESHER \\ 
Random doors param & 0.5 \\ 
\hline
	\end{tabular}
\end{center}

\begin{center}
	\begin{tabular}{ | c | c | c | c | c | }
\hline
Tam. habs. & Modelos & Instancias/modelo & Total habs. & Tiempo \\ \hline 
6 & 20 & 1 & 20 & 0.0395 \\ 
6 & 20 & 2 & 40 & 0.2505 \\ 
6 & 30 & 1 & 30 & 0.1479 \\ 
6 & 30 & 2 & 60 & 0.9846 \\ 
6 & 40 & 1 & 40 & 0.4307 \\ 
6 & 40 & 2 & 80 & 2.9426 \\ 
6 & 50 & 1 & 50 & 0.9980 \\ 
6 & 50 & 2 & 100 & 7.6388 \\ 
\hline
	\end{tabular}
\end{center}

bs-opt-variabilty-sample-fixed
\begin{center}
	\begin{tabular}{ | c | c | }
\hline
 		Property & Value \\ \hline
DoorGenType & RANDOM \\ 
SolverType & BestSearch \\ 
BestSearch DPE Divisor & 0.5 \\ 
CacheType & ALWAYS \\ 
Random doors param & 0.3 \\ 
\hline
	\end{tabular}
\end{center}

\begin{center}
	\begin{tabular}{ | c | c | c | c | c | }
\hline
Tam. habs. & Modelos & Instancias/modelo & Total habs. & Tiempo \\ \hline 
6 & 50 & 2 & 100 & 1.6345 \\ 
\hline
	\end{tabular}
\end{center}


















\section{Comparación ejemplo optimizado}

bs-opt-compare
\begin{center}
	\begin{tabular}{ | c | c | }
\hline
 		Property & Value \\ \hline
DoorGenType & RANDOM \\ 
Refresher cache divisor & 10 \\ 
SolverType & BestSearch \\ 
BestSearch DPE Divisor & 0.9 \\ 
CacheType & REFRESHER \\ 
Random doors param & 0.5 \\ 
\hline
	\end{tabular}
\end{center}


bs-nocache-compare
\begin{center}
	\begin{tabular}{ | c | c | }
\hline
 		Property & Value \\ \hline
DoorGenType & ALL \\ 
SolverType & BestSearch \\ 
BestSearch DPE Divisor & 1.0 \\ 
CacheType & NO CACHE \\ 
\hline
	\end{tabular}
\end{center}


bs-alwayscache-compare
\begin{center}
	\begin{tabular}{ | c | c | }
\hline
 		Property & Value \\ \hline
DoorGenType & ALL \\ 
SolverType & BestSearch \\ 
BestSearch DPE Divisor & 1.0 \\ 
CacheType & ALWAYS \\ 
\hline
	\end{tabular}
\end{center}


\begin{center}
	\begin{tabular}{ | c | c | c | c | c | c | c | }
\hline
Tam. h. & Modelos & Inst./modelo & Total & NoCache & Always & Opt \\ \hline 
8 & 4 & 2 & 8 & 0.0333 & 0.0174 & 0.0030 \\ 
8 & 4 & 4 & 16 & 0.2818 & 0.0649 & 0.0123 \\ 
8 & 4 & 6 & 24 & 1.2533 & 0.1949 & 0.0354 \\ 
8 & 4 & 8 & 32 & 3.3727 & 0.4231 & 0.0780 \\ 
8 & 6 & 2 & 12 & 0.1696 & 0.0666 & 0.0107 \\ 
8 & 6 & 4 & 24 & 1.7436 & 0.2856 & 0.0481 \\ 
8 & 6 & 6 & 36 & 8.6428 & 1.0331 & 0.1302 \\ 
8 & 6 & 8 & 48 & 19.8587 & 2.7444 & 0.3155 \\ 
8 & 8 & 2 & 16 & 0.5825 & 0.2523 & 0.0210 \\ 
8 & 8 & 4 & 32 & 7.3917 & 1.1891 & 0.1030 \\ 
8 & 8 & 6 & 48 & 33.7262 & 3.6202 & 0.3438 \\ 
8 & 8 & 8 & 64 & 107.0144 & 9.6228 & 0.8999 \\ 
\hline
	\end{tabular}
\end{center}




\subsection{Impacto de la generación de puertas aleatoria}

\begin{center}
	\begin{tabular}{ | c | c | }
\hline
 		Property & Value \\ \hline
DoorGenType & RANDOM \\ 
SolverType & BestSearch \\ 
BestSearch DPE Divisor & 1.0 \\ 
CacheType & NO CACHE \\ 
\hline
	\end{tabular}
\end{center}

\begin{center}
	\begin{tabular}{ | c | c | c | c | c | c | c | }
\hline
Tam. habs. & Modelos & Inst./modelo & Total & 0.8f & 0.6f & 0.4f \\ \hline 
10 & 4 & 8 & 32 & 4.6552 & 3.3017 & 1.7974 \\ 
10 & 4 & 10 & 40 & 9.8173 & 8.2366 & 4.1857 \\ 
10 & 6 & 8 & 48 & 29.2931 & 20.9267 & 15.4530 \\ 
10 & 6 & 10 & 60 & 69.2747 & 57.3447 & 34.6857 \\ 
\hline
	\end{tabular}
\end{center}


\subsection{Impacto del Caché Refresher}

\begin{center}
	\begin{tabular}{ | c | c | }
\hline
 		Property & Value \\ \hline
DoorGenType & ALL \\ 
SolverType & BestSearch \\ 
BestSearch DPE Divisor & 1.0 \\ 
CacheType & REFRESHER \\ 
\hline
	\end{tabular}
\end{center}

\begin{center}
	\begin{tabular}{ | c | c | c | c | c | c | c | c |}
\hline
Tam. habs. & Modelos & Inst./modelo & Total & N=2 & N=5 & N=10 \\ \hline 
6 & 4 & 10 & 40 & 1.3768 & 1.0284 & 0.7653 \\ 
6 & 4 & 15 & 60 & 6.7213 & 5.5230 & 3.4248 \\ 
6 & 6 & 10 & 60 & 8.7300 & 6.7349 & 5.5666 \\ 
6 & 6 & 15 & 90 & 43.7862 & 35.1948 & 26.3091 \\ 
\hline
	\end{tabular}
\end{center}


   % 
\chapter{Trabajo futuro.}\label{cap:capitulo8}

Comentaremos algunas posibles mejoras para el sistema, enfocadas principalmente a una posible versión comerciable del sistema que ofrezca algunas características extra y añadan atractivo para potenciar su posible comercialización.




\section{Mejora del editor de habitaciones.}

La implementación del editor de habitaciones es bastante rudimentaria. Podría haberse hecho más amigable al usuario, pero para un esbozo del sistema era suficiente. Además, normalmente, en los mapas de tiles se emplean varias capas para añadir detalles. Se podría añadir esta característica para que los usuarios pudieran añadir capas extra en las habitaciones.

\section{Mapa de tamaño autoajustable}

En el enunciado del problema, se supone que el mapa es lo suficientemente grande como para albergar cualquier distribución de habitaciones. Para solventar este requerimiento, se ha empleado un mapa de tiles grande, pero sería más correcto utilizar una implementación que vaya autoaumentando el tamaño de la matriz que representa el mapa conforme se vaya necesitando más espacio para albergar habitaciones.

\section{Fitness extra}

Se ha implementado un sistema lo suficientemente flexible como para añadir otros objetivos de guía en la construcción del mapa. Esto da al usuario la ventaja de poder elaborar sus propios objetivos.

Podría añadirse dos objetivos de \emph{horizontalidad} y \emph{verticalidad}. Estos objetivos darían más puntuación a un movimiento, si este promueve que la disposición final sea horizontal o vertical. El cómputo numérico de este objetivo podría hacerse computando la recta horizontal o vertical media, y calculando la suma de las desviaciones de todas las habitaciones con respecto a dicha recta

Otra medida objetivo a tener encuenta podría ser una medida que estuviera relacionada con la dispersión o condensación de la disposición de las habitaciones. Esto podría realizarse calculando la desviación típica con respecto a la posición media de todas las habitaciones, obteniendo así una medida representativa de esta característica.

\section{Flexibilidad en elección de puertas potenciales}

Otro componente interesante a añadir en el sistema sería uno dedicado a la afinación de la elección de puertas potenciales. Por ejemplo, para un plataformas, podemos querer que las puertas potenciales sean solamente horizontales y no verticales. Quizá para un cierto tipo de juego, solo nos interese cierto subconjunto de puertas potenciales, y no todas, de forma que podríamos añadir un selector de puertas al editor de habitaciones y emplear dicho subconjunto elegido por el usuario en la ejecución del sistema.

\section{Portar a móvil}

Debido a los tiempos de ejecución obtenidos en la experimentación y a los distintos componentes relacionados con la eficiencia del tiempo de ejecución, sería posible portar el sistema a móviles para el uso en juegos en estos dispositivos. Un elemento importante en este aspecto es el denominado \emph{fitness caché}, que nos permite sacrificar calidad de la generación a cambio de mejores tiempo de eficiencia.

\section{Otros fitness caché}

Relacionado con la sección anterior, podríamos elaborar varias implementaciones de la interfaz del fitness caché que controlaran mediante parámetro la eficiencia en cuanto a tiempo de ejecución del sistema.

Una posibilidad sería cachear solamente cada N intentos. Así, podríamos configurar este parámetro N para calibrar la relación calidad/tiempo de ejecución del sistema. Se podría hacer también con una probabilidad de que se cachee o no. Con esto, tendríamos bastante flexibilidad a la hora de ajustar la eficiencia al dispositivo en el que corra el sistema.

Otra posibilidad sería regenerar la caché cada N pasos del algoritmo. De esta forma, podemos controlar la relación calidad/tiempo de ejecución de forma más homogénea para todos los posibles movimientos.

\section{Backtracking con guardado de movimientos}

Como se comentó en el capítulo anterior, el sistema no está dotado de backtracking ya que no se vio en necesidad de ello, pero podría implementarse. Esto es gracias a que guardamos la lista de movimientos que construye una solución para poder obtener el mapa de tiles asociado a dicha solución. Así, cada N pasos, podríamos guardar la lista de movimientos generados para poder reanudar la generación en ese punto. También podríamos simplemente tener en cuenta si hay varias soluciones mejores, es decir, soluciones que están dotadas de la misma puntuación, o dando un cierto margen de diferencia. De esta forma, también podríamos reanudar la generación a partir de ellas.
   % 
\begin{thebibliography}{99}

\addcontentsline{toc}{chapter}{Bibliografía.}

\bibitem{lucasarts} \href{https://es.wikipedia.org/wiki/LucasArts}{LucasArts Wikipedia page}

\bibitem{lod} \href{http://www.techopedia.com/definition/11791/level-of-detail-lod}{Techopedia LOD page}

\bibitem{voxel} Voxel wikipedia

\bibitem{quadtree} Quadtree wikipedia

\bibitem{octree} Octree wikipedia

\bibitem{brush} Brush wikipedia

\bibitem{quake} Quake wikipedia

\bibitem{rlike} RogueLike wikipedia

\bibitem{costgames} \href{https://en.wikipedia.org/wiki/List_of_most_expensive_video_games_to_develop}{List of most expensive video games to develop.}

\bibitem{gdroles1} \href{http://creativeskillset.org/creative_industries/games/job_roles}{Gamedev Job Roles at CreativeSkillSet}

\bibitem{gdroles2} \href{https://en.wikipedia.org/wiki/Video_game_development#Roles}{Video Game Development Wikipedia page}

\end{thebibliography}









%%  Apendices
%%%%%%%%%%%%%%%%%%%%%%%%%%%%%%%%%%%%%%%%%%%%%%%%%%%%%%%%%%%%%%%%%%%%%%

%\appendix
%\include{apendice}  % Puedo poner la ontología OSMV y la arquitectura de OW

%%  Bibliografia
%%%%%%%%%%%%%%%%%%%%%%%%%%%%%%%%%%%%%%%%%%%%%%%%%%%%%%%%%%%%%%%%%%%%%%
\newpage
\addcontentsline{toc}{chapter}{Bibliografía}
\bibliographystyle{alpha}
\bibliography{biblio/bibliografia.tex}

\end{document}


\documentclass[a4paper,12pt]{book}

%********************* PAQUETES ADICIONALES *******************

% Para poder escribir acentos tal cual
%\usepackage{isolatin1}
\usepackage[utf8]{inputenc}
\usepackage{fancyvrb}

\newcommand{\caja}[1]{
 \makebox[8mm]{\rule[-1.5ex]{0pt}{6ex}\scriptsize #1}}

\newcommand{\cajaa}[1]{
 \makebox[18mm]{\rule[-1.5ex]{0pt}{6ex}\scriptsize #1}}

\usepackage{shortvrb}
%\MakeShortVerb{\ß}
% \DefineShortVerb{\ß}
%\MakeShortVerb{\ß}

% Para disponer de m·s sÌmbolos matem·ticos
\usepackage{amsfonts,amssymb,latexsym,fancyhdr}

% Para que las Ûrdenes como \chapter o \tableofcontents generen 
% los tÌtulos en espaÒol.
\usepackage[spanish,activeacute]{babel}

% Una extensiÛn para la definiciÛn de nuevos teoremas
\usepackage{theorem}

\theorembodyfont{\normalfont}
\newtheorem{defi}{{\sc Definición}}
\newtheorem{teo}{{\sc Teorema}}


%*********************** FORMATO DE P¡GINA **************************

\voffset          -5mm
\cfoot{}

\pagestyle{fancy}
\fancyhead{} 
\fancyhead[RE]{{\small  \nouppercase{\leftmark}}}
\fancyhead[LO]{{\small  \nouppercase{\rightmark}}}
\fancyhead[LE,RO]{\thepage}


%==============================================================================
% Cajas numeradas para incluir cÛdigo
%==============================================================================

\newlength{\hsbw}

\newcounter{sesioncount}
\setcounter{sesioncount}{0}

%%%%%%%%%%%%%%%%%%% caja no numerada  %%%%%%%%%%%%%%%%%%%%%%%%%

\newenvironment{sesion*}
  {\begin{flushleft}
   \setlength{\hsbw}{\linewidth}
   \addtolength{\hsbw}{-\arrayrulewidth}
   \addtolength{\hsbw}{-\tabcolsep}
   \addtolength{\hsbw}{-\tabcolsep}
   \begin{tabular}{@{}|c@{}|@{}}\hline
   \begin{minipage}[b]{\hsbw}
   \vspace{2mm}
   \begingroup}
  {\endgroup\vspace{-3mm}
   \end{minipage} \\
   \hline
   \end{tabular}
   \end{flushleft}}


%%%%%%%%%%%%%%%%% caja normal numerada  %%%%%%%%%%%%%%%%%%%%%%
\newenvironment{sesion}
  {\begin{flushleft}
   \setlength{\hsbw}{\linewidth}
   \begin{tabular}{@{}|c@{}|@{}}\hline
   \begin{minipage}[b]{\hsbw}
   \vspace*{4mm}
   \em
   \begingroup}
  {\endgroup \vspace{4mm}
   \end{minipage} \\
   \hline
   \end{tabular}
   \end{flushleft}}


%%%%%%%%%%%%%%%%%% caja abierta por debajo %%%%%%%%%%%%%%
\newenvironment{sesion+}
  {\begin{flushleft}
 % \refstepcounter{sesioncount}
   \setlength{\hsbw}{\linewidth}
   \addtolength{\hsbw}{-\arrayrulewidth}
   \addtolength{\hsbw}{-\tabcolsep}
   \addtolength{\hsbw}{-\tabcolsep}
   \begin{tabular}{@{}|c@{}|@{}}\hline
   \begin{minipage}[b]{\hsbw}
   \vspace*{-.5pt}
   \begin{flushright}
   \rule{0.01in}{.15in}\rule{0.3in}{0.01in}\hspace{-0.35in}
   \raisebox{0.04in}{\makebox[0.3in][c]{\footnotesize \thesesioncount}}
                                       % Si pongo \footnotesize\it
                                       % el numerito sale inclinado
   \end{flushright}
   \vspace*{-.47in}
   \begingroup}
  {\endgroup\vspace{-3mm}
   \end{minipage} \\
   %\hline
   \end{tabular}
   \end{flushleft}}


%%%%%%%%%%%% caja abierta por arriba %%%%%%%%%%%%%%%%%%%%%%%%%
\newenvironment{sesion-}
  {\begin{flushleft}
   \setlength{\hsbw}{\linewidth}
   \addtolength{\hsbw}{-\arrayrulewidth}
   \addtolength{\hsbw}{-\tabcolsep}
   \addtolength{\hsbw}{-\tabcolsep}
   \begin{tabular}{@{}|c@{}|@{}}%\hline
   \begin{minipage}[b]{\hsbw}
   \vspace{2mm}
   \begingroup}
  {\endgroup\vspace{-3mm}
   \end{minipage} \\
   \hline
   \end{tabular}
   \end{flushleft}}

\newenvironment{sesionx}
  {\begin{flushleft}
   \setlength{\hsbw}{\linewidth}
   \addtolength{\hsbw}{-\arrayrulewidth}
   \addtolength{\hsbw}{-\tabcolsep}
   \addtolength{\hsbw}{-\tabcolsep}
   \begin{tabular}{@{}|c@{}|@{}}%\hline
   \begin{minipage}[b]{\hsbw}
   \vspace{2mm}
   \begingroup}
  {\endgroup\vspace{-3mm}
   \end{minipage} \\
   %\hline
   \end{tabular}
   \end{flushleft}}



\newcommand{\boxref}[1]
  {\fbox{\footnotesize \phantom{4}\!\!\!\!\!\! \ref{#1}}}
%Si pongo \footnotesize\it
%el numerito sale inclinado

\newcommand{\boxrefdos}[2]
  {\fbox{\footnotesize \phantom{4}\!\!\!\!\!\! \ref{#1}-\ref{#2}}}
%Si pongo \footnotesize\it
%el numerito sale inclinado

\newcommand{\codeindex}[1]{\index{#1@{\tt #1}}}

\newcommand{\labelindex}[1]{\label{#1}\codeindex{#1}}


%%%%%%%%%%%%%%%%%%%%%%%%%%%%%%%%%%%%%%%%%%%%%%%%%%%%%%%%%%%%%%%%%%%%%%%%%%%%%%%
%% ß Verbatim extendido                                                      %%
%%%%%%%%%%%%%%%%%%%%%%%%%%%%%%%%%%%%%%%%%%%%%%%%%%%%%%%%%%%%%%%%%%%%%%%%%%%%%%%

\DefineVerbatimEnvironment{CVerbatim}{Verbatim}
{ commandchars=\\\{\},
  codes={\catcode`$=3\catcode`$=3},
  fontsize=\small,
  fontfamily=verdana,
  frame=single}


%=======================================================================

%=======================================================================


%\renewcommand{\baselinestretch}{1.3}  % espacio y algo 
\parskip 1ex 
%\renewcommand{\_}{\bold} 
\newcommand{\ol}{\overline} 
\renewcommand{\>}{\vec} 
\newcommand{\lan}{\langle} 
\newcommand{\ran}{\rangle} 
 
%LETRAS GRIEGAS 
\newcommand{\gga}{\alpha} 
\newcommand{\ggb}{\beta} 
\newcommand{\ggD}{\Delta} 
\newcommand{\ggd}{\delta} 
\newcommand{\gget}{\eta} 
\newcommand{\gge}{\varepsilon} 
\newcommand{\ggf}{\varphi} 
\newcommand{\ggF}{\Phi} 
\renewcommand{\ggg}{\gamma} 
\newcommand{\ggG}{\Gamma} 
\newcommand{\ggk}{\kappa} 
\newcommand{\ggl}{\lambda} 
\newcommand{\ggm}{\mu} 
\newcommand{\ggn}{\nu} 
\newcommand{\ggps}{\psi} 
\newcommand{\ggP}{\Pi} 
\newcommand{\ggp}{\pi} 
\newcommand{\ggr}{\rho} 
\newcommand{\ggS}{\Sigma} 
\newcommand{\ggs}{\sigma} 
\newcommand{\ggta}{\tau} 
\newcommand{\ggT}{\Theta} 
\newcommand{\ggt}{\theta} 
\newcommand{\ggo}{\omega} 
\newcommand{\ggO}{\Omega} 
 
% LETRAS CALIGRAFICAS 
\newcommand{\C}[1]{{\cal #1}}
\newcommand{\U}{{\cal U}} 
\newcommand{\W}{{\cal W}} 
\newcommand{\K}{{\cal K}} 
 
% SIMBOLOS LOGICOS 
\renewcommand{\le}{\exists} 
\newcommand{\la}{\forall} 
\newcommand{\lif}{\to} 
\newcommand{\liff}{\leftrightarrow}
\newcommand{\Rif}{\Leftarrow} 
\newcommand{\rif}{\leftarrow}  
\newcommand{\Lor}{\bigvee} 
\newcommand{\Land}{\bigwedge} 
 
% SIMBOLOS METALOGICOS 
\newcommand{\Lif}{\quad \Longrightarrow \quad} 
\newcommand{\Lifn}{\Longrightarrow} 
\newcommand{\Liff}{\quad \Longleftrightarrow \quad} 
\newcommand{\If}{\Longrightarrow} 
\newcommand{\Lifr}{\quad \Longleftarrow \quad} 
\newcommand{\Lifrn}{\Longleftarrow} 
\newcommand{\Liffn}{\Longleftrightarrow} 
\newcommand{\direc}{\noindent \fbox{$\Longrightarrow$} \quad} 
\newcommand{\recip}{\noindent \fbox{$\Longleftarrow$} \quad} 
\newcommand{\impd}[2]{\noindent \fbox{\parbox{1.1in}{(#1) $\Lif$ (#2)}} 
                       \newline} 
\newcommand{\Life}{\quad \mid\!\Longrightarrow \quad} 
\newcommand{\Lifen}{\mid\!\Longrightarrow} 
\newcommand{\Lifre}{\quad \Longleftarrow\!\mid \quad} 
\newcommand{\Lifren}{\Longleftarrow\!\mid} 
\newcommand{\Lifa}{\quad \Uparrow \quad} 
\newcommand{\Lifan}{\Uparrow} 
\newcommand{\Lifae}{\underline{\Uparrow}} 
\newcommand{\Liffa}{\quad \Updownarrow \quad} 
\newcommand{\Liffan}{\Updownarrow} 
\newcommand{\Lifab}{\quad \Downarrow \quad} 
\newcommand{\Lifabn}{\Downarrow} 
%\newcommand{\nLif}{\quad /\!\!\!\!\!\!\!\Longrightarrow } 
\newcommand{\nLif}{ \ /\!\!\!\!\!\!\!\Longrightarrow} 
\newcommand{\Liffp}{\Leftrightarrow} 
\newcommand{\Lifp}{\mid\!\Rightarrow} 
\newcommand{\Rlifp}{\Leftarrow\!\mid} 
 
% SECUENCIAS 
\newcommand{\veca}[2]{#1_1 \dots #1_#2}            %$veca xn$ --> x_1... x_n 
\newcommand{\vecb}[2]{#1_1, \dots, #1_#2}          %$vecb xn$ --> x_1,..., x_n 
\newcommand{\vecc}[2]{(#1_1, \dots, #1_#2)}        %$vecc xn$ --> (x_1,..., x_n) 
 
\newcommand{\sust}[4]{#1_{\vecb #2 #4} [\vecb #3 #4]} 
        %$sust a x t n$ --> a_{x_1,...,x_n}[t_1,...,t_n] 
\newcommand{\lr}[2]{\lan #1,#2 \ran} 
 
% CONJUNTOS 
\newcommand{\co}{\subseteq} 
\newcommand{\coe}{\subset} 
\newcommand{\pot}[2]{{#1}^{\textstyle #2}} 
\newcommand{\ale}[1]{\aleph_{#1}} 
\newcommand{\kkk}[1]{\stackrel{#1}{\smile}} 
 
 
 
% MODELOS 
\newcommand{\m}{\models} 
\newcommand{\te}{\vdash} 
\newcommand{\inm}{\; \widetilde \subset \;} %inmersion (en infijo) 
\newcommand{\inme}[2]{\C {#1} \inm \C {#2}} %inmersion 
\newcommand{\sel}{\; \prec \;} %subestructura elemental (en infijo) 
\newcommand{\sele}[2]{\C {#1} \sel \C {#2}} %subestructura elemental 
\newcommand{\iel}{\; \widetilde \prec \;} %inmersion elemental (en infijo) 
\newcommand{\iele}[2]{\C {#1} \iel \C {#2}} %inmersion elemental 
\renewcommand{\hom}{\; \simeq \;} 
\newcommand{\homo}[2]{\C {#1} \hom \C {#2}} %homomorfismo 
\newcommand{\iso}{\; \cong \;}                   %subestructura 
\newcommand{\isom}[2]{\C {#1} \iso \C {#2}} %isomorfismo 
\newcommand{\sub}{\; \coe \;} 
\newcommand{\sube}[2]{\C {#1} \sub \C {#2}} %subestructura 
 
% FIN DE LINEA 
\newcommand{\nl}{\newline} 
 
% FLECHAS 
\newcommand{\lra}{\longrightarrow} 
\newcommand{\Lra}{\Longrightarrow} 
\newcommand{\lla}{\longleftarrow} 
\newcommand{\Lla}{\Longleftarrow} 
\newcommand{\llra}{\longleftrightarrow} 
\newcommand{\Llra}{\Longleftrightarrow} 
 
 
\newcommand{\dip}{\searrow\!\!\!\!\nwarrow} 
\newcommand{\dis}{\nearrow\!\!\!\!\swarrow} 
\newcommand{\udp}{\nwarrow} 
\newcommand{\ddp}{\searrow} 
\newcommand{\uds}{\nearrow} 
\newcommand{\dds}{\swarrow} 
 
\newcommand{\ua}{\uparrow} 
\newcommand{\da}{\downarrow} 
 
 
\newcommand{\uda}{\uparrow\!\!\downarrow} 

%\makeatletter
%\def\@begintheorem#1#2{\it \trivlist \item[\hskip \labelsep{\em {\bf
%#2. #1}}] }
%\makeatother

\usepackage{cmtt}
\renewcommand{\ttdefault}{cmtt}



%%% Local Variables: 
%%% mode: latex
%%% TeX-master: "dea"
%%% End: 

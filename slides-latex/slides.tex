\documentclass[12]{beamer}
\usepackage{beamerthemesplit}
\usepackage[spanish]{babel}
\usepackage[latin1]{inputenc}
\usepackage{libertine}
\usepackage{color}
\usepackage{graphicx}
\begin{document}
\title{Investigaci�n en la generaci�n de mapas para videojuegos.}
\author{Alejandro Segu� D�az}
\date{\today}

\frame{\titlepage}

\frame{\frametitle{�ndice}\tableofcontents}


\section{Videojuegos}
\frame{\frametitle{�Por qu� desarrollo de videojuegos?}
	\begin{itemize}
		\item Mercado lider indiscutible a nivel internacional\pause
		\item Conveniente investigaci�n en este campo\pause
		\item En ocasiones implica desarrollo en otros campos
	\end{itemize}
}

\section{Objetivo}
\frame{\frametitle{Contexto}
	\begin{itemize}
		\item Muchos g�neros de videojuegos requieren mapas en los que dar rienda suelta al jugador para recorrer, investigar y disfrutar\pause
		\item Creaci�n de mapas consume recursos horas/persona\pause
		\item Empresas grandes dedican equipos a esta tarea
	\end{itemize}
}

\frame{\frametitle{Problema}
	\begin{itemize}
		\item Recientemente han aparecido equipos denominados \emph{indies}, caracterizados por tener bajo personal y presupuesto\pause
		\item Emerge necesidades relacionadas con cubrir esta falta de recursos \pause
		\item Abordaremos el problema de ahorrar tiempo en la creaci�n de mapas
	\end{itemize}
}

\frame{\frametitle{Soluci�n}
	\begin{itemize}
		\item Elaborando un sistema que genere mapas, cubrimos necesidades \pause
		\item Cada juego tiene sus propias reglas \pause
		\item El sistema elaborado debe adaptarse a dichas reglas \pause
		\item Este trabajo se ha elaborado en funci�n de requisitos impuestos por\pause
	\end{itemize}
\begin{figure}[]
\centering
\includegraphics[width=50px]{tgk.jpg}
\caption{A simple caption \label{overflow}}
\end{figure}
}



\end{document}
